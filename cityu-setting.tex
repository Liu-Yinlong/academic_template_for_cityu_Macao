\graphicspath{{Images/}{./}} % Specifies where to look for included images (trailing slash required)


\usepackage[dvipsnames]{xcolor}
\usepackage{tikz}
\usepackage{pgfplots}
\pgfplotsset{compat=1.18}
\usepackage{graphicx}
\usepackage{booktabs} % Allows the use of \toprule, \midrule and \bottomrule for better rules in tables
\usepackage{ctex}%中文支持
\usepackage{fontspec}
\usepackage{tcolorbox}



% 更和谐的学术配色方案
\definecolor{primaryBlue}{RGB}{55, 95, 145} % 深蓝色 - 主要结构元素
\definecolor{secondaryBlue}{RGB}{85, 125, 165} % 中蓝色 - 次要元素
\definecolor{accentGray}{RGB}{80, 90, 100} % 深灰色 - 强调元素
\definecolor{lightGray}{RGB}{240, 245, 250} % 浅灰色 - 背景


%----------------------------------------------------------------------------------------
%	SELECT LAYOUT THEME
%----------------------------------------------------------------------------------------

% Beamer comes with a number of default layout themes which change the colors and layouts of slides. Below is a list of all themes available, uncomment each in turn to see what they look like.

%\usetheme{default}
%\usetheme{AnnArbor}
%\usetheme{Antibes}
%\usetheme{Bergen}
%\usetheme{Berkeley}
\usetheme{Berlin}
%\usetheme{Boadilla}
%\usetheme{CambridgeUS}
%\usetheme{Copenhagen}
%\usetheme{Darmstadt}
%\usetheme{Dresden}
%\usetheme{Frankfurt}
%\usetheme{Goettingen}
%\usetheme{Hannover}
%\usetheme{Ilmenau}
%\usetheme{JuanLesPins}
%\usetheme{Luebeck}
%\usetheme{Madrid}
%\usetheme{Malmoe}
%\usetheme{Marburg}
%\usetheme{Montpellier}
%\usetheme{PaloAlto}
%\usetheme{Pittsburgh}
%\usetheme{Rochester}
%\usetheme{Singapore}
%\usetheme{Szeged}
%\usetheme{Warsaw}

%----------------------------------------------------------------------------------------
%	SELECT COLOR THEME
%----------------------------------------------------------------------------------------

% Beamer comes with a number of color themes that can be applied to any layout theme to change its colors. Uncomment each of these in turn to see how they change the colors of your selected layout theme.

%\usecolortheme{albatross}
%\usecolortheme{beaver}
%\usecolortheme{beetle}
%\usecolortheme{crane}
%\usecolortheme{dolphin}
%\usecolortheme{dove}
%\usecolortheme{fly}
%\usecolortheme{lily}
%\usecolortheme{monarca}
%\usecolortheme{seagull}
%\usecolortheme{seahorse}
%\usecolortheme{spruce}
%\usecolortheme{whale}
%\usecolortheme{wolverine}


%%%-------颜色设置
% 应用主题颜色
\colorlet{cityuTheme}{PineGreen}
\usecolortheme[named=cityuTheme]{structure}

% 为各元素设置颜色
% \setbeamercolor{title}{fg=white, bg=primaryBlue}
% \setbeamercolor{frametitle}{fg=white, bg=primaryBlue}
% \setbeamercolor{section in head/foot}{fg=white, bg=primaryBlue}
% \setbeamercolor{subsection in head/foot}{fg=white, bg=secondaryBlue}
% \setbeamercolor{block title}{fg=white, bg=primaryBlue}
% \setbeamercolor{block body}{fg=black, bg=lightGray}
% \setbeamercolor{alerted text}{fg=accentGray}
% \setbeamercolor{itemize item}{fg=primaryBlue}
% \setbeamercolor{enumerate item}{fg=primaryBlue}
% \setbeamercolor{example text}{fg=secondaryBlue}

% 定制tcolorbox样式
% \tcbset{
%     colback=lightGray,
%     colframe=primaryBlue,
%     arc=2mm,
%     boxrule=0.5pt
% }
%-------

%----------------------------------------------------------------------------------------
%	SELECT FONT THEME & FONTS
%----------------------------------------------------------------------------------------

% Beamer comes with several font themes to easily change the fonts used in various parts of the presentation. Review the comments beside each one to decide if you would like to use it. Note that additional options can be specified for several of these font themes, consult the beamer documentation for more information.

%\usefonttheme{default} % Typeset using the default sans serif font
%\usefonttheme{serif} % Typeset using the default serif font (make sure a sans font isn't being set as the default font if you use this option!)
%\usefonttheme{structurebold} % Typeset important structure text (titles, headlines, footlines, sidebar, etc) in bold
%\usefonttheme{structureitalicserif} % Typeset important structure text (titles, headlines, footlines, sidebar, etc) in italic serif
%\usefonttheme{structuresmallcapsserif} % Typeset important structure text (titles, headlines, footlines, sidebar, etc) in small caps serif
%\usefonttheme{professionalfonts}%更改数学字体
\usefonttheme[onlymath]{serif}
%------------------------------------------------

%\usepackage{mathptmx} % Use the Times font for serif text
%\usepackage{palatino} % Use the Palatino font for serif text

%\usepackage{helvet} % Use the Helvetica font for sans serif text
%\usepackage[default]{opensans} % Use the Open Sans font for sans serif text
%\usepackage[default]{FiraSans} % Use the Fira Sans font for sans serif text
%\usepackage[default]{lato} % Use the Lato font for sans serif text
\usepackage{tgheros} %使用好看的字体
\usepackage[T1]{fontenc}


%配置中文字体

\newCJKfontfamily\heiti[Path=./CJKfonts/]{MiSans-Regular.otf}  % 黑体族
%可以在此处添加自己喜欢的字体
\setCJKsansfont[
				Path = ./CJKfonts/,
				Extension = .otf,
				UprightFont = *-Regular,
				BoldFont = *-Bold,
				]{MiSans}

%----------------------------------------------------------------------------------------
%	SELECT INNER THEME
%----------------------------------------------------------------------------------------

% Inner themes change the styling of internal slide elements, for example: bullet points, blocks, bibliography entries, title pages, theorems, etc. Uncomment each theme in turn to see what changes it makes to your presentation.

%\useinnertheme{default}
%\useinnertheme{circles}
%\useinnertheme{rectangles}
%\useinnertheme{rounded}
%\useinnertheme{inmargin}

%----------------------------------------------------------------------------------------
%	SELECT OUTER THEME
%----------------------------------------------------------------------------------------

% Outer themes change the overall layout of slides, such as: header and footer lines, sidebars and slide titles. Uncomment each theme in turn to see what changes it makes to your presentation.


%\useoutertheme{default}
%\useoutertheme{infolines}
%\useoutertheme{miniframes}
%\useoutertheme{smoothbars}
%\useoutertheme{sidebar}
%\useoutertheme{split}
%\useoutertheme{shadow}
%\useoutertheme{tree}
%\useoutertheme{smoothtree}

%\setbeamertemplate{footline} % Uncomment this line to remove the footer line in all slides
%\setbeamertemplate{footline}[page number] % Uncomment this line to replace the footer line in all slides with a simple slide count

\setbeamertemplate{navigation symbols}{} % Remove navigation symbols for a cleaner look


% 设置章节标题栏颜色
%\setbeamercolor{section in head/foot}{fg=white, bg=Green!40!black}

% 设置子节标题栏颜色
%\setbeamertemplate{subsection in head/foot}{}




